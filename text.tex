
\documentclass[a4paper, 12pt]{article}

\usepackage[a4paper, left=15mm, right=15mm, top=10mm, bottom=20mm]{geometry}

\usepackage[protrusion=true,expansion=true,final]{microtype}
\usepackage{paratype}
\renewcommand{\familydefault}{\sfdefault}
\usepackage[catalan]{babel}
\usepackage[utf8]{inputenc}
\usepackage[T1]{fontenc}


%Definició de la ela geminada per tal que accepti el punt volat del teclat
\def\xgem{%
    \ifmmode
        \csname normal@char\string"\endcsname l%
    \else
        \leftllkern=0pt\rightllkern=0pt\raiselldim=0pt
        \setbox0\hbox{l}\setbox1\hbox{l\/}\setbox2\hbox{.}%
        \advance\raiselldim by \the\fontdimen5\the\font
        \advance\raiselldim by -\ht2
        \leftllkern=-.25\wd0%
        \advance\leftllkern by \wd1
        \advance\leftllkern by -\wd0
        \rightllkern=-.25\wd0%
        \advance\rightllkern by -\wd1
        \advance\rightllkern by \wd0
        \allowhyphens\discretionary{-}{}%
        {\kern\leftllkern\raise\raiselldim\hbox{.}%
        \kern\rightllkern}\allowhyphens
    \fi
}
\def\Xgem{%
    \ifmmode
        \csname normal@char\string"\endcsname L%
    \else
        \leftllkern=0pt\rightllkern=0pt\raiselldim=0pt
        \setbox0\hbox{L}\setbox1\hbox{L\/}\setbox2\hbox{.}%
        \advance\raiselldim by .5\ht0
        \advance\raiselldim by -.5\ht2
        \leftllkern=-.125\wd0%
        \advance\leftllkern by \wd1
        \advance\leftllkern by -\wd0
        \rightllkern=-\wd0%
        \divide\rightllkern by 6
        \advance\rightllkern by -\wd1
        \advance\rightllkern by \wd0
        \allowhyphens\discretionary{-}{}%
        {\kern\leftllkern\raise\raiselldim\hbox{.}%
        \kern\rightllkern}\allowhyphens
    \fi
}
\DeclareTextCommand{\textperiodcentered}{T1}[1]{%
    \ifnum\spacefactor=998
        \Xgem
    \else
        \xgem
    \fi#1}

% Imatges
\usepackage{graphicx}
\graphicspath{{./pic/}}


\newcommand\blfootnote[1]{%
  \begingroup
  \renewcommand\thefootnote{}\footnote{#1}%
  \addtocounter{footnote}{-1}%
  \endgroup
}


% Sagnats
\usepackage{parskip}
\frenchspacing
\linespread{1.25}

\usepackage{hyperref}
\hypersetup {
    hidelinks,
    pdfauthor    = {Carlos Luna Mota},
    pdfsubject   = {Recomanacions d'ampliació de matemàtiques},
    pdftitle     = {Ampliació de Matemàtiques},
    pdfkeywords  = {ampliació, matemàtiques},
    pdfcreator   = {LaTeX},
    pdfproducer  = {pdflatex}
}
\urlstyle{same}


\title{Ampliació de matemàtiques per alumnes d'institut}
\date{}
\author{Carlos Luna Mota}

\begin{document}

    \maketitle


    \vspace{-2ex}
    \blfootnote{Versió: \today}Les matemàtiques que fem a l'institut són, en general, les eines bàsiques que tothom necessita per funcionar en societat. No seria just dedicar-hi més hores a ampliar aquest temari perquè no tothom necessita aquesta ampliació i perquè cal dedicar aquest temps a altres assignatures per tal de tenir una formació prou completa i variada.

    Malgrat tot, no és estrany trobar alumnes que, per gust o amb la vista posada en uns futurs estudis tècnics, voldrien dominar millor aquesta matèria i troben a faltar aprofundir-hi més. Tradicionalment, aquest aprofundiment es duia a terme mitjançant assignatures optatives o classes de reforç, però avui dia hi ha moltes més alternatives i en aquest document en faig un recull de les que acostumo a recomanar als meus alumnes.

    Abans de començar, però, cal tenir en compte que cada alumne és un món i que no es tracta de fer tot el que aquí recomano, sinó de trobar, entre totes aquestes recomanacions, algun tema que motivi a l'alumne i que s'adapti al seu nivell i preferències.


    \section*{Activitats recomanades}

        \subsection*{Presencials}

            \begin{itemize}
                \item Visites al \href{https://mmaca.cat/}{\textbf{Museu de Matemàtiques de Catalunya}} o a alguna de les seves exposicions itinerants.
                \item Participació en competicions com les \href{https://www.cangur.org/}{\textbf{Proves Cangur}}, l'\href{https://www.cangur.org/olimpiades/}{\textbf{Olimpíada Matemàtica}}, el \href{https://fm.feemcat.org/}{\textbf{Fem mates}}/\href{https://sites.google.com/view/mesmates/inici}{\textbf{+mates}} o els \href{https://feemcat.org/category/activitats-alumnes/esprint/}{\textbf{Problemes a l'esprint}}.
                \item Participació en programes d'aprofundiment, com ara \href{https://feemcat.org/bojos/}{\textbf{Bojos per la matemàtica}} o \href{http://www.estalmat.cat/}{\textbf{Estalmat}}.
                \item Participació en concursos de fotografía matemàtica d'\href{https://ademgi.feemcat.org/multimatge/}{\textbf{ADEMGI}} o l'\href{https://fotografiamatematica.cat/}{\textbf{ABEAM}}.
                \item Activitats d'oci que requereixin fer servir el pensament lògic, com ara un club d'\href{https://escacs.cat/}{\textbf{escacs}} o les \href{https://ca.wikipedia.org/wiki/Joc_d\%27escapada_en_viu}{\textbf{\emph{scape rooms}}}.
            \end{itemize}

        \subsection*{En línia}

            \begin{itemize}
                \item Els cursos en línia de \href{https://brilliant.org/courses/}{\textbf{Brilliant}} o \href{https://es.khanacademy.org/}{\textbf{Khan Academy}}.
                \item Seguir blogs com \href{https://www.gaussianos.com/}{\textbf{Gaussianos}} o \href{http://divermates.es/blog/}{\textbf{Divermates}}, canals de Youtube com \href{https://www.youtube.com/channel/UCoxcjq-8xIDTYp3uz647V5A}{\textbf{Numberphile}} o \href{https://www.youtube.com/channel/UCH-Z8ya93m7_RD02WsCSZYA}{\textbf{Derivando}} o usuaris de xarxes socials com la \href{https://twitter.com/Cshearer41}{\textbf{Catriona Agg}} o en \href{https://twitter.com/jamestanton}{\textbf{James Tanton}}.
            \end{itemize}

    \newpage %%%%%%%%%%%%%%%%%%%%%%%%%%%%%%%%%%%%%%%%%%%%%%%%%%%%%%%%%%%%%%%%%%%

    \section*{Lectures recomanades}

        \subsection*{Divulgació matemàtica}

            \begin{itemize}
                \item Qualsevol llibre d'en \href{https://ca.wikipedia.org/wiki/Martin_Gardner}{\textbf{Martin Gardner}} és un petit tresor d'idees matemàtiques. L'\href{https://ca.wikipedia.org/wiki/Ian_Stewart_(matem\%C3\%A0tic)}{\textbf{Ian Stewart}} va recollir el seu llegat i més recentment també destaquen molt l'\href{https://en.wikipedia.org/wiki/Alex_Bellos}{\textbf{Alex Bellos}} i l'\href{https://es.wikipedia.org/wiki/Adri\%C3\%A1n_Paenza}{\textbf{Adrián Paenza}}. De tots aquests autors podeu trobar llibres traduïts (sovint només al castellà) a les llibreries. D'en Paenza, a més, podeu trobar tots els llibres en format PDF a \href{http://cms.dm.uba.ar/material/paenza}{\textbf{la seva web}}.
                \item A nivell nacional, la \href{https://ca.wikipedia.org/wiki/Clara_Grima_Ruiz}{\textbf{Clara Grima}} i en \href{http://claudialsina.com/ca/publicacions}{\textbf{Claudi Alsina}} tenen diversos llibres de divulgació matemàtica que poden resultar interessants per alumnes d'edats diverses.
                \item L'editorial RBA publica periòdicament col·leccions com ara \textbf{El Mundo Es Matemático} o \textbf{Desafíos Matemáticos}. La qualitat no és uniforme però són llibres entretinguts i donen una idea general força acurada dels temes dels que parlen. L'editorial Planeta publica llibres de divulgació matemàtica en la seva línia \href{https://www.planetadelibros.com/coleccion-drakontos/0000965600}{\textbf{Drakontos}} i l'editorial Nivola té una molt bona col·lecció de biografies matemàtiques anomenada \href{https://www.nivola.com/listado_libros.php?idcol=2&nombre=4\%20-\%20La\%20matem\%E1tica\%20en\%20sus\%20personajes\%20-\%20Biograf\%EDas\%20de\%20los\%20grandes\%20matem\%E1ticos}{\textbf{La matemática en sus personajes}}.
            \end{itemize}

        \subsection*{Enigmes}

            \begin{itemize}
                \item Els llibres clàssics d'enigmes matemàtics van ser escrits per en \href{https://ca.wikipedia.org/wiki/Sam_Loyd}{\textbf{Sam Loyd}} i en \href{https://ca.wikipedia.org/wiki/Henry_Dudeney}{\textbf{Henry Dudeney}}. L'\href{https://ca.wikipedia.org/wiki/\%C3\%89douard_Lucas}{\textbf{Édouard Lucas}} és potser menys conegut, simplement, per escriure en francès en comptes de en anglès. En tots tres casos, la seva fama és més que merescuda.
                \item Avui dia destaquen autors com \href{https://es.wikipedia.org/wiki/Carlo_Frabetti}{\textbf{Carlo Fabretti}}, \href{https://en.wikipedia.org/wiki/Dennis_Shasha}{\textbf{Dennis Shasha}} o \href{https://ca.wikipedia.org/wiki/Raymond_Smullyan}{\textbf{Raymond Smullyan}} i, a nivell nacional, cal seguir la pista d'en \href{https://gedisa.com/autor.aspx?codaut=1298}{\textbf{Jordi Deulofeu}} i en \href{https://www.casadellibro.com/libros-ebooks/miquel-capo-dolz/128544}{\textbf{Miquel Capó}}
                \item L'editorial Gedisa té dues col·leccions excel·lents de llibres d'enigmes: \href{https://www.gedisa.com/articulos.aspx?modo=c&fam=040}{\textbf{Juegos}} i \href{https://www.gedisa.com/articulos.aspx?modo=c&fam=1024}{\textbf{Desafios Matemáticos}}. Allà trobareu traduïts bona part dels autors que acabo de citar i molts altres igualment interessants.
            \end{itemize}

        \subsection*{Literatura matemàtica}

            \begin{itemize}
                \item El primer nom que ve al cap quan un parla de literatura matemàtica és el de \href{https://ca.wikipedia.org/wiki/Lewis_Carroll}{\textbf{Lewis Carroll}}, amb les seves \emph{Alícies}, però en \href{https://ca.wikipedia.org/wiki/Edwin_Abbott_Abbott}{\textbf{Robert Abbott}}, amb el seu \emph{Planilàndia}, també és un clàssic recomanable.
                \item Malgrat tot, jo crec que alguns relats de \href{https://ca.wikipedia.org/wiki/Jorge_Luis_Borges}{\textbf{Jorge Luis Borges}}, com ara \emph{la Biblioteca de Babel} o \emph{el Libro de Arena}, són millors exponents del gènere i tenen molta més qualitat literària.
                \item A un nivell molt més avançat, la ciència ficció de \href{https://es.wikipedia.org/wiki/Greg_Egan}{\textbf{Greg Egan}} sovint fa servir matemàtiques universitàries com a fil conductor.
            \end{itemize}
        
    \newpage %%%%%%%%%%%%%%%%%%%%%%%%%%%%%%%%%%%%%%%%%%%%%%%%%%%%%%%%%%%%%%%%%%%

    \section*{Jocs recomanats}

        \subsection*{D'enginy}

            \begin{itemize}
                \item Qualsevol \href{http://robspuzzlepage.com/mainmenu.htm}{\textbf{joc d'enginy}}, especialment aquells de natura geomètrica, com el \href{https://ca.wikipedia.org/wiki/Tangram}{\textbf{Tangram}} o el \href{https://ca.wikipedia.org/wiki/Cub_Soma}{\textbf{Cub Soma}}, serà un bon exercici per treballar matemàtiques a un nivell \emph{tàctil}. De tangrams, per exemple, n'hi ha de molts tipus, i al MMACA han desenvolupat un de nou, el \href{https://github.com/CarlosLunaMota/The-Egyptian-Tangram}{\textbf{Tangram Egipci}}, que està ple de propietats matemàtiques.
                \item Altres jocs d'enginy clàssics inclouen el \href{https://en.wikipedia.org/wiki/T_puzzle}{\textbf{Puzzle de la T}}, els \href{https://ca.wikipedia.org/wiki/Pent\%C3\%B2mino}{\textbf{Pentòminos}}, les \href{https://ca.wikipedia.org/wiki/Torres_de_Hanoi}{\textbf{Torres de Hanoi}}, el \href{https://ca.wikipedia.org/wiki/Solitari_(joc_de_tauler)}{\textbf{Solitari}} o el famós \href{https://ca.wikipedia.org/wiki/Cub_de_Rubik}{\textbf{Cub de Rubik}} i les seves mil variants.
                \item Molts d'aquests jocs els trobareu a fires d'artesania, la \href{https://mmaca.cat/botiga/}{\textbf{botiga del MMACA}} i l'\href{https://www.abacus.coop/ca/home}{\textbf{Abacus}}. Amb material que tingueu per casa, com ara una baralla de cartes, és fàcil formar \href{https://ca.wikipedia.org/wiki/Quadrat_m\%C3\%A0gic}{\textbf{Quadrats Màgics}} i \href{https://ca.wikipedia.org/wiki/Quadrat_grecollat\%C3\%AD}{\textbf{Quadrats Grecollatins}} que tenen un munt de propietats i variants per explorar.
            \end{itemize}

        \subsection*{De taula}

            \begin{itemize}
                \item Els jocs de taula d'estratègia són una bona manera de treballar la concentració, la lògica i la resolució de problemes. Estem parlant dels \href{https://ca.wikipedia.org/wiki/Escacs}{\textbf{Escacs}}, és clar, però també de les \href{https://ca.wikipedia.org/wiki/Dames}{\textbf{Dames}} i de jocs que són menys coneguts aquí però que tenen una gran repercussió en altres cultures, com ara el \href{https://ca.wikipedia.org/wiki/Go}{\textbf{Go}}, el \href{https://ca.wikipedia.org/wiki/Sh\%C5\%8Dgi}{\textbf{Shogi}} o l'\href{https://ca.wikipedia.org/wiki/Aual\%C3\%A9}{\textbf{Aualé}}. Aquest darrer joc, tant popular al continent africà, és una bona manera d'exercitar el càlcul mental, com també ho és el jugar al \href{https://ca.wikipedia.org/wiki/Backgammon}{\textbf{Backgammon}}.
                \item De caire més actual, tenim jocs com el \href{https://boardgamegeek.com/boardgame/13/catan}{\textbf{Catan}} i el \href{https://boardgamegeek.com/boardgame/822/carcassonne}{\textbf{Carcassonne}} que no tenen res a envejar als clàssics a nivell d'estratègia però, si volem triar jocs de taula moderns que tinguin una forta component matemàtica, us recomano fixar-vos en el \href{https://boardgamegeek.com/boardgame/1198/set}{\textbf{Set}}, el \href{https://boardgamegeek.com/boardgame/681/quarto}{\textbf{Quarto}}, el \href{https://boardgamegeek.com/boardgame/1038/tantrix}{\textbf{Tantrix}}, el \href{https://boardgamegeek.com/boardgame/2655/hive}{\textbf{Hive}}, el \href{https://boardgamegeek.com/boardgame/6931/katamino}{\textbf{Katamino}}, el \href{https://boardgamegeek.com/boardgame/63268/spot-it}{\textbf{Dobble}} i el \href{https://boardgamegeek.com/boardgame/204583/kingdomino}{\textbf{Kingdomino}}.
                \item D'alguns jocs clàssics de paper i llapis, com ara el \href{https://ca.wikipedia.org/wiki/Quadrats}{\textbf{Dots \& Boxes}} o l'\href{https://ca.wikipedia.org/wiki/Hex_(joc)}{\textbf{Hex}}, se n'han publicat llibres sencers d'anàlisi matemàtic. De fet, hi ha tota una branca de la matemàtica dedicada a l'estudi d'aquests \emph{jocs abstractes} i a Portugal cada any fan un \href{http://ludicum.org/cnjm}{\textbf{Campionat Nacional}} juvenil amb alguns dels millors jocs d'aquesta categoria. Si voleu jocs de taula abstractes de qualitat us recomano les marques \href{https://www.gigamic.com/}{\textbf{Gigamic}} i \href{https://www.steffen-spiele.de/}{\textbf{Stephen Spiele}} i, a nivell nacional, \href{https://nestorgames.com}{\textbf{Nestorgames}}.
            \end{itemize}

        \subsection*{Digitals}

            \begin{itemize}
                \item Avui dia potser és més fàcil jugar amb un dispositiu mòbil que no pas en persona i podeu trobar versions digitals de tots (o gairebé tots) els jocs proposats a l'apartat anterior. També és fàcil trobar aplicacions plenes de passatemps clàssics. La \href{https://play.google.com/store/apps/details?id=name.boyle.chris.sgtpuzzles}{\textbf{Simon Tatham's Portable Puzzle Collection}} és probablement una de les millors i més completes.
                \item Els dispositius mòbils permeten treballar amb objectes virtuals complexos, com demostren jocs com \href{https://play.google.com/store/apps/details?id=com.ustwo.monumentvalley}{\textbf{Monument Valley}}, on cal resoldre trencaclosques en un món de geometria peculiar.
                \item Altres aplicacions destacables, a nivell didàctic, serien l'\href{https://play.google.com/store/apps/details?id=com.hil_hk.euclidea}{\textbf{Euclidea}} i la \href{https://play.google.com/store/apps/details?id=com.hil_hk.pythagorea}{\textbf{Pitagorea}}.
            \end{itemize}

    \newpage %%%%%%%%%%%%%%%%%%%%%%%%%%%%%%%%%%%%%%%%%%%%%%%%%%%%%%%%%%%%%%%%%%%
    
    \section*{Altres recomanacions}

        \subsection*{Lectura}

            Sovint, els problemes de comprensió lectora agreugen qualsevol dificultat que l'alumne pugui tenir a l'hora de fer matemàtiques.

            Practicar la lectura no només no va en contra d'una bona formació tècnica sinó que sovint n'és la clau. Cal, però, fugir de lectures obligatòries i trobar gèneres que motivin a l'alumne, com ara la divulgació científica (\href{https://ca.wikipedia.org/wiki/Bill_Bryson}{\textbf{Bill Bryson}}, \href{https://ca.wikipedia.org/wiki/Pere_Estupiny\%C3\%A0}{\textbf{Pere Estupinyà}}, \href{https://en.wikipedia.org/wiki/William_Poundstone}{\textbf{William Poundstone}}, \dots) o la ciència ficció (\href{https://ca.wikipedia.org/wiki/Philip_K._Dick}{\textbf{Philip K. Dick}}, \href{https://ca.wikipedia.org/wiki/Isaac_Asimov}{\textbf{Isaac Asimov}}, \href{https://ca.wikipedia.org/wiki/Arthur_C._Clarke}{\textbf{Arthur C. Clarke}}, \dots).

        \subsection*{Idiomes}

            Malauradament, encara avui dia resulta difícil trobar cert tipus de material en català (o fins i tot en castellà) i cal recórrer a llibres, pàgines web i vídeos en anglès. Aprendre idiomes en general, i anglès en particular, ens obre la porta a moltíssim material imprès i en línia que d'altra manera ens quedaria fora de l'abast.

            Si hagués de triar una única ``recomanació alternativa'' d'aquesta llista, probablement seria la d'aprendre anglès. Malgrat tot, el mètode per fer-ho no necessàriament passa per cursos i acadèmies (tot i que probablement siguin la manera més ràpida i efectiva de fer-ho) i potser resulta més motivador, simplement, començar a \textbf{veure sèries en anglès}, \textbf{traduir les nostres cançons preferides} o fer servir alguna aplicació gamificada de l'estil \href{https://play.google.com/store/apps/details?id=com.duolingo}{\textbf{Duolingo}}.

        \subsection*{Art}

            L'art i la matemàtica són dues disciplines que sovint s'han fet servir mútuament com a font d'inspiració. Més enllà de la literatura matemàtica, de la que ja hem parlat anteriorment, podem trobar moltes matemàtiques a la música de \href{https://ca.wikipedia.org/wiki/Johann_Sebastian_Bach}{\textbf{J. S. Bach}}, a l'arquitectura de \href{https://ca.wikipedia.org/wiki/Buckminster_Fuller}{\textbf{B. Fuller}} i a la pintura de \href{https://ca.wikipedia.org/wiki/Maurits_Cornelis_Escher}{\textbf{M. C. Escher}}.

            L'\href{https://ca.wikipedia.org/wiki/Origami}{\textbf{origami}} i el \href{https://en.wikipedia.org/wiki/Kirigami}{\textbf{kirigami}} són dues disciplines artístiques d'origen oriental que es fonamenten, respectivament, en el plegat i el tall de fulls de paper. Actualment, totes dues disciplines viuen una segona edat d'or gràcies a la matemàtica, que hi ha sabut trobar nombroses propietats i aplicacions pràctiques (des del plegat de panells solars fins a la creació de nano-màquines). A nivell clàssic puc recomanar els llibres de \href{https://www.casadellibro.com/libros-ebooks/vicente-palacios/60484}{\textbf{Vicente Palacios}} i \href{https://en.wikipedia.org/wiki/John_Montroll}{\textbf{John Montroll}} i d'autors més contemporanis recomano els llibres (i documentals) de \href{http://www.origamiheaven.com/}{\textbf{David Mitchell}}, \href{https://en.wikipedia.org/wiki/Erik_Demaine}{\textbf{Erik D. Demaine}} i \href{https://en.wikipedia.org/wiki/Robert_J._Lang}{\textbf{Robert J. Lang}}.

            El món de l'espectacle també té la seva ració de matemàtiques de la mà de la \href{https://ca.wikipedia.org/wiki/Matem\%C3\%A0gia}{\textbf{matemàgia}}. Aquí els referents serien en \href{http://www.ehu.eus/~mtpalezp/}{\textbf{Pedro Alegría}}, en \href{http://fblasco.net/}{\textbf{Fernando Blasco}} i en \href{https://magiaymatematicas.blogspot.com/}{\textbf{Sergio Belmonte}}. D'autor internacionals destacaria en \href{https://ca.wikipedia.org/wiki/Martin_Gardner}{\textbf{Martin Gardner}} i en \href{https://en.wikipedia.org/wiki/Roberto_Giobbi}{\textbf{Roberto Giobbi}}.

        \subsection*{Programació}

            Programar ordinadors (o qualsevol aparell electrònic) és una tasca enriquidora que té molts punts en comú amb la resolució de problemes matemàtics. En contra del que molta gent creu, programar ordinadors no és especialment difícil. La major dificultat sovint rau en trobar un bon curs/tutorial en la nostra llengua (\href{https://www.sololearn.com/Course/Python/}{\textbf{en anglès és molt més fàcil}}). Per alumnes sense experiència prèvia, el llenguatge de programació \href{https://ca.wikipedia.org/wiki/Python}{\textbf{Python}} ofereix una de les portes d'entrada més accessibles a aquest món i, a diferència d'altres llenguatges educatius (com ara l'\href{https://ca.wikipedia.org/wiki/Scratch_(llenguatge_de_programaci\%C3\%B3)}{\emph{Scratch}} o el \href{https://ca.wikipedia.org/wiki/Llenguatge_de_programaci\%C3\%B3_Logo}{\emph{Logo}}), es tracta d'un llenguatge potent i amb força sortida professional.

            Un cop familiaritzats amb els rudiments bàsics, serà fàcil trobar projectes de programació que cridin l'atenció de l'alumne, com ara dibuixar \href{https://ca.wikipedia.org/wiki/Fractal}{\textbf{fractals}} amb la \href{https://opentechschool.github.io/python-beginners/es_CL/simple_drawing.html}{\textbf{Tortuga de Python}}, programar jocs de taula senzills amb \href{https://www.pygame.org/}{\textbf{PyGame}}, fer un simulador del \href{https://ca.wikipedia.org/wiki/Joc_de_la_vida}{\textbf{Joc de la Vida}} de Conway en format text, \dots

            Si el tema agrada i es vol progressar encara més, webs com \href{https://www.topcoder.com/community/arena}{\textbf{Top Coder}}, \href{https://www.codewars.com/}{\textbf{Code Wars}}, \href{https://projecteuler.net/}{\textbf{Project Euler}} o \href{https://jutge.org/}{\textbf{Jutge.org}} obren la porta a la programació competitiva, un autèntic repte per a gent de totes les edats.

        \subsection*{Esport}

            Fer algun tipus d'\textbf{activitat física} de manera regular (especialment a l'aire lliure) pot ajudar a millorar el rendiment acadèmic de l'alumne: millora la circulació sanguínia i l'oxigenació del cervell, redueix els nivells d'estrès i ansietat i, sobretot, trenca la rutina d'hores d'estudi i pantalles. Al llarg de la història són molts els matemàtics que han trobat la inspiració que els calia per resoldre un problemes mentre passejaven (vegeu, per exemple, el descobriment dels \href{https://en.wikipedia.org/wiki/Hamilton_Walk}{\textbf{quaternions}}).



\end{document}
